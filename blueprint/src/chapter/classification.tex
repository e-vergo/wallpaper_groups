\chapter{Classification Theorem}

This chapter proves the main theorem: every wallpaper group is isomorphic to exactly one of the 17 groups defined in the previous chapter.

%%%%%%%%%%%%%%%%%%%%%%%%%%%%%%%%%%%%%%%%%%%%%%%%%%%%%
\section{Verification: Each Group is a Wallpaper Group}
%%%%%%%%%%%%%%%%%%%%%%%%%%%%%%%%%%%%%%%%%%%%%%%%%%%%%

\begin{lemma}[p1 is a wallpaper group]
    \label{lem:p1_is_wallpaper}
    \lean{WallpaperGroup.p1.isWallpaperGroup}
    \uses{def:p1, def:wallpaper_group}
    \notready
    The group p1 is a wallpaper group.
\end{lemma}

\begin{proof}
    \uses{lem:lattice_discrete, lem:lattice_cocompact}
    \notready
    p1 is the translation group of a lattice, which is discrete and cocompact.
\end{proof}

\begin{lemma}[p2 is a wallpaper group]
    \label{lem:p2_is_wallpaper}
    \lean{WallpaperGroup.p2.isWallpaperGroup}
    \uses{def:p2, def:wallpaper_group}
    \notready
    The group p2 is a wallpaper group.
\end{lemma}

\begin{lemma}[pm is a wallpaper group]
    \label{lem:pm_is_wallpaper}
    \lean{WallpaperGroup.pm.isWallpaperGroup}
    \uses{def:pm, def:wallpaper_group}
    \notready
    The group pm is a wallpaper group.
\end{lemma}

\begin{lemma}[pg is a wallpaper group]
    \label{lem:pg_is_wallpaper}
    \lean{WallpaperGroup.pg.isWallpaperGroup}
    \uses{def:pg, def:wallpaper_group}
    \notready
    The group pg is a wallpaper group.
\end{lemma}

\begin{lemma}[cm is a wallpaper group]
    \label{lem:cm_is_wallpaper}
    \lean{WallpaperGroup.cm.isWallpaperGroup}
    \uses{def:cm, def:wallpaper_group}
    \notready
    The group cm is a wallpaper group.
\end{lemma}

\begin{lemma}[pmm is a wallpaper group]
    \label{lem:pmm_is_wallpaper}
    \lean{WallpaperGroup.pmm.isWallpaperGroup}
    \uses{def:pmm, def:wallpaper_group}
    \notready
    The group pmm is a wallpaper group.
\end{lemma}

\begin{lemma}[pmg is a wallpaper group]
    \label{lem:pmg_is_wallpaper}
    \lean{WallpaperGroup.pmg.isWallpaperGroup}
    \uses{def:pmg, def:wallpaper_group}
    \notready
    The group pmg is a wallpaper group.
\end{lemma}

\begin{lemma}[pgg is a wallpaper group]
    \label{lem:pgg_is_wallpaper}
    \lean{WallpaperGroup.pgg.isWallpaperGroup}
    \uses{def:pgg, def:wallpaper_group}
    \notready
    The group pgg is a wallpaper group.
\end{lemma}

\begin{lemma}[cmm is a wallpaper group]
    \label{lem:cmm_is_wallpaper}
    \lean{WallpaperGroup.cmm.isWallpaperGroup}
    \uses{def:cmm, def:wallpaper_group}
    \notready
    The group cmm is a wallpaper group.
\end{lemma}

\begin{lemma}[p4 is a wallpaper group]
    \label{lem:p4_is_wallpaper}
    \lean{WallpaperGroup.p4.isWallpaperGroup}
    \uses{def:p4, def:wallpaper_group}
    \notready
    The group p4 is a wallpaper group.
\end{lemma}

\begin{lemma}[p4m is a wallpaper group]
    \label{lem:p4m_is_wallpaper}
    \lean{WallpaperGroup.p4m.isWallpaperGroup}
    \uses{def:p4m, def:wallpaper_group}
    \notready
    The group p4m is a wallpaper group.
\end{lemma}

\begin{lemma}[p4g is a wallpaper group]
    \label{lem:p4g_is_wallpaper}
    \lean{WallpaperGroup.p4g.isWallpaperGroup}
    \uses{def:p4g, def:wallpaper_group}
    \notready
    The group p4g is a wallpaper group.
\end{lemma}

\begin{lemma}[p3 is a wallpaper group]
    \label{lem:p3_is_wallpaper}
    \lean{WallpaperGroup.p3.isWallpaperGroup}
    \uses{def:p3, def:wallpaper_group}
    \notready
    The group p3 is a wallpaper group.
\end{lemma}

\begin{lemma}[p3m1 is a wallpaper group]
    \label{lem:p3m1_is_wallpaper}
    \lean{WallpaperGroup.p3m1.isWallpaperGroup}
    \uses{def:p3m1, def:wallpaper_group}
    \notready
    The group p3m1 is a wallpaper group.
\end{lemma}

\begin{lemma}[p31m is a wallpaper group]
    \label{lem:p31m_is_wallpaper}
    \lean{WallpaperGroup.p31m.isWallpaperGroup}
    \uses{def:p31m, def:wallpaper_group}
    \notready
    The group p31m is a wallpaper group.
\end{lemma}

\begin{lemma}[p6 is a wallpaper group]
    \label{lem:p6_is_wallpaper}
    \lean{WallpaperGroup.p6.isWallpaperGroup}
    \uses{def:p6, def:wallpaper_group}
    \notready
    The group p6 is a wallpaper group.
\end{lemma}

\begin{lemma}[p6m is a wallpaper group]
    \label{lem:p6m_is_wallpaper}
    \lean{WallpaperGroup.p6m.isWallpaperGroup}
    \uses{def:p6m, def:wallpaper_group}
    \notready
    The group p6m is a wallpaper group.
\end{lemma}

%%%%%%%%%%%%%%%%%%%%%%%%%%%%%%%%%%%%%%%%%%%%%%%%%%%%%
\section{Distinctness}
%%%%%%%%%%%%%%%%%%%%%%%%%%%%%%%%%%%%%%%%%%%%%%%%%%%%%

\begin{definition}[Wallpaper group invariants]
    \label{def:invariants}
    \lean{WallpaperGroupInvariants}
    \uses{def:wallpaper_point_group, def:wallpaper_translations, def:symmorphic}
    \notready
    The following are isomorphism invariants of a wallpaper group:
    \begin{enumerate}
        \item The isomorphism class of the point group $\PG(\Gamma)$
        \item The lattice type (determined by the symmetry group of $\TG(\Gamma)$)
        \item Whether $\Gamma$ is symmorphic
        \item The location of rotation centers relative to the lattice
        \item The orientation of reflection axes relative to the lattice
    \end{enumerate}
\end{definition}

\begin{lemma}[The 17 groups are pairwise distinct]
    \label{lem:distinct}
    \lean{wallpaperGroups_distinct}
    \uses{def:invariants, def:p1, def:p2, def:pm, def:pg, def:cm, def:pmm, def:pmg, def:pgg, def:cmm, def:p4, def:p4m, def:p4g, def:p3, def:p3m1, def:p31m, def:p6, def:p6m}
    \notready
    The 17 wallpaper groups p1, p2, pm, pg, cm, pmm, pmg, pgg, cmm, p4, p4m, p4g, p3, p3m1, p31m, p6, p6m are pairwise non-isomorphic.
\end{lemma}

\begin{proof}
    \notready
    Distinguish groups by their invariants:
    \begin{itemize}
        \item Point group order distinguishes many: orders 1, 2, 3, 4, 6, 8, 12
        \item Within same point group: lattice type or symmorphic property differs
        \item p3m1 vs p31m: both have $\Dih{3}$ and hexagonal lattice, but differ in whether 3-fold centers lie on reflection axes
    \end{itemize}
\end{proof}

%%%%%%%%%%%%%%%%%%%%%%%%%%%%%%%%%%%%%%%%%%%%%%%%%%%%%
\section{Completeness by Case Exhaustion}
%%%%%%%%%%%%%%%%%%%%%%%%%%%%%%%%%%%%%%%%%%%%%%%%%%%%%

\begin{lemma}[Oblique lattice completeness]
    \label{lem:oblique_complete}
    \lean{obliqueLattice_wallpaperGroups}
    \uses{def:oblique_lattice, lem:compatible_point_groups, lem:wallpaper_ses}
    \notready
    The only wallpaper groups with an oblique lattice are p1 and p2.
\end{lemma}

\begin{proof}
    \uses{lem:p1_is_wallpaper, lem:p2_is_wallpaper}
    \notready
    The only point groups compatible with an oblique lattice are $\Cyc{1}$ and $\Cyc{2}$. Both are cyclic, so the short exact sequence always splits. This gives exactly p1 and p2.
\end{proof}

\begin{lemma}[Rectangular lattice completeness]
    \label{lem:rectangular_complete}
    \lean{rectangularLattice_wallpaperGroups}
    \uses{def:rectangular_lattice, lem:compatible_point_groups}
    \notready
    The wallpaper groups with a rectangular lattice are pm, pg, pmm, pmg, pgg.
\end{lemma}

\begin{proof}
    \uses{lem:pm_is_wallpaper, lem:pg_is_wallpaper, lem:pmm_is_wallpaper, lem:pmg_is_wallpaper, lem:pgg_is_wallpaper, lem:non_symmorphic_glide}
    \notready
    Compatible point groups: $\Cyc{1}, \Cyc{2}, \Dih{1}, \Dih{2}$.
    \begin{itemize}
        \item $\Cyc{1}$: gives p1 (but that's oblique, so excluded)
        \item $\Cyc{2}$: gives p2 (also oblique type)
        \item $\Dih{1}$: split → pm, non-split → pg
        \item $\Dih{2}$: split → pmm, one direction non-split → pmg, both non-split → pgg
    \end{itemize}
\end{proof}

\begin{lemma}[Centered rectangular lattice completeness]
    \label{lem:centered_complete}
    \lean{centeredRectangularLattice_wallpaperGroups}
    \uses{def:centered_rectangular_lattice, lem:compatible_point_groups}
    \notready
    The wallpaper groups with a centered rectangular lattice are cm and cmm.
\end{lemma}

\begin{proof}
    \uses{lem:cm_is_wallpaper, lem:cmm_is_wallpaper}
    \notready
    Compatible point groups: $\Dih{1}, \Dih{2}$ (excluding $\Cyc{1}, \Cyc{2}$ which give oblique-type groups). The centering forces the extension to split, giving cm and cmm.
\end{proof}

\begin{lemma}[Square lattice completeness]
    \label{lem:square_complete}
    \lean{squareLattice_wallpaperGroups}
    \uses{def:square_lattice, lem:compatible_point_groups}
    \notready
    The wallpaper groups with a square lattice are p4, p4m, p4g.
\end{lemma}

\begin{proof}
    \uses{lem:p4_is_wallpaper, lem:p4m_is_wallpaper, lem:p4g_is_wallpaper}
    \notready
    Only $\Cyc{4}$ and $\Dih{4}$ require a square lattice.
    \begin{itemize}
        \item $\Cyc{4}$: always splits → p4
        \item $\Dih{4}$: split → p4m, non-split → p4g
    \end{itemize}
\end{proof}

\begin{lemma}[Hexagonal lattice completeness]
    \label{lem:hexagonal_complete}
    \lean{hexagonalLattice_wallpaperGroups}
    \uses{def:hexagonal_lattice, lem:compatible_point_groups}
    \notready
    The wallpaper groups with a hexagonal lattice are p3, p3m1, p31m, p6, p6m.
\end{lemma}

\begin{proof}
    \uses{lem:p3_is_wallpaper, lem:p3m1_is_wallpaper, lem:p31m_is_wallpaper, lem:p6_is_wallpaper, lem:p6m_is_wallpaper}
    \notready
    Point groups $\Cyc{3}, \Cyc{6}, \Dih{3}, \Dih{6}$ require hexagonal lattice.
    \begin{itemize}
        \item $\Cyc{3}$: splits → p3
        \item $\Cyc{6}$: splits → p6
        \item $\Dih{3}$: splits, but two distinct orientations → p3m1 and p31m
        \item $\Dih{6}$: splits → p6m
    \end{itemize}
\end{proof}

%%%%%%%%%%%%%%%%%%%%%%%%%%%%%%%%%%%%%%%%%%%%%%%%%%%%%
\section{Main Theorem}
%%%%%%%%%%%%%%%%%%%%%%%%%%%%%%%%%%%%%%%%%%%%%%%%%%%%%

\begin{theorem}[Classification of wallpaper groups]
    \label{thm:classification}
    \lean{wallpaper_classification}
    \uses{def:wallpaper_group, lem:distinct, lem:oblique_complete, lem:rectangular_complete, lem:centered_complete, lem:square_complete, lem:hexagonal_complete, lem:point_group_crystallographic, thm:bravais_classification}
    \notready
    Every wallpaper group is isomorphic to exactly one of the following 17 groups:
    \[
        \text{p1, p2, pm, pg, cm, pmm, pmg, pgg, cmm, p4, p4m, p4g, p3, p3m1, p31m, p6, p6m.}
    \]
\end{theorem}

\begin{proof}
    \notready
    \textbf{Existence}: Lemmas~\ref{lem:p1_is_wallpaper}--\ref{lem:p6m_is_wallpaper} show each of the 17 is a wallpaper group.

    \textbf{Distinctness}: Lemma~\ref{lem:distinct}.

    \textbf{Completeness}: Let $\Gamma$ be any wallpaper group. By Lemma~\ref{lem:point_group_crystallographic}, $\PG(\Gamma)$ is one of the 10 crystallographic point groups. By Theorem~\ref{thm:bravais_classification}, $\TG(\Gamma)$ is one of 5 lattice types. The compatibility constraints (Lemma~\ref{lem:compatible_point_groups}) and extension enumeration (Lemmas~\ref{lem:oblique_complete}--\ref{lem:hexagonal_complete}) show $\Gamma$ is isomorphic to one of the 17.
\end{proof}

\begin{corollary}[There are exactly 17 wallpaper groups]
    \label{cor:seventeen}
    \lean{wallpaper_count}
    \uses{thm:classification}
    \notready
    Up to isomorphism, there are exactly 17 wallpaper groups.
\end{corollary}

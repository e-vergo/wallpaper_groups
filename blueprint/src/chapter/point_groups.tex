\chapter{Point Groups}

This chapter classifies the finite subgroups of $\Orth(2)$. These are the ``point groups'' that will appear as quotients of wallpaper groups.

%%%%%%%%%%%%%%%%%%%%%%%%%%%%%%%%%%%%%%%%%%%%%%%%%%%%%
\section{Cyclic and Dihedral Subgroups}
%%%%%%%%%%%%%%%%%%%%%%%%%%%%%%%%%%%%%%%%%%%%%%%%%%%%%

\begin{definition}[Cyclic point group]
    \label{def:cyclic_point_group}
    \lean{CyclicPointGroup}
    \uses{def:rotation_matrix}
    \notready
    For $n \geq 1$, the \emph{cyclic point group} $\Cyc{n} \subset \Orth(2)$ is the subgroup generated by the rotation $R_{2\pi/n}$:
    \[
        \Cyc{n} = \langle R_{2\pi/n} \rangle = \{ R_{2\pi k/n} \mid k = 0, 1, \ldots, n-1 \}.
    \]
    This is the group of rotational symmetries of a regular $n$-gon.
\end{definition}

\begin{lemma}[Cyclic group order]
    \label{lem:cyclic_order}
    \lean{CyclicPointGroup.card}
    \uses{def:cyclic_point_group}
    \notready
    $|\Cyc{n}| = n$.
\end{lemma}

\begin{proof}
    \notready
    $R_{2\pi/n}$ has order exactly $n$ since $(R_{2\pi/n})^k = R_{2\pi k/n} = I$ iff $n \mid k$.
\end{proof}

\begin{definition}[Dihedral point group]
    \label{def:dihedral_point_group}
    \lean{DihedralPointGroup}
    \uses{def:cyclic_point_group, def:reflection_matrix}
    \notready
    For $n \geq 1$, the \emph{dihedral point group} $\Dih{n} \subset \Orth(2)$ is generated by $R_{2\pi/n}$ and the reflection $S_0$:
    \[
        \Dih{n} = \langle R_{2\pi/n}, S_0 \rangle.
    \]
    This is the group of all symmetries (rotations and reflections) of a regular $n$-gon.
\end{definition}

\begin{lemma}[Dihedral group order]
    \label{lem:dihedral_order}
    \lean{DihedralPointGroup.card}
    \uses{def:dihedral_point_group}
    \notready
    $|\Dih{n}| = 2n$.
\end{lemma}

\begin{proof}
    \uses{lem:cyclic_order}
    \notready
    $\Dih{n}$ contains $n$ rotations (from $\Cyc{n}$) and $n$ reflections $\{S_0, S_{2\pi/n}, \ldots, S_{2\pi(n-1)/n}\}$.
\end{proof}

\begin{lemma}[Dihedral group structure]
    \label{lem:dihedral_structure}
    \lean{DihedralPointGroup.semidirectProduct}
    \uses{def:dihedral_point_group, def:cyclic_point_group}
    \notready
    $\Dih{n} \cong \Cyc{n} \sdp \Cyc{2}$ where $\Cyc{2}$ acts on $\Cyc{n}$ by inversion.
\end{lemma}

\begin{proof}
    \notready
    The map $\Cyc{n} \to \Dih{n}$ is injective with index 2. Conjugation by $S_0$ satisfies $S_0 R_\theta S_0 = R_{-\theta}$, giving the inversion action.
\end{proof}

\begin{lemma}[Dihedral embedding from Mathlib]
    \label{lem:dihedral_embedding}
    \lean{DihedralPointGroup.equivDihedralGroup}
    \uses{def:dihedral_point_group}
    \notready
    There is a group isomorphism between $\Dih{n} \subset \Orth(2)$ and \texttt{DihedralGroup n} from Mathlib.
\end{lemma}

%%%%%%%%%%%%%%%%%%%%%%%%%%%%%%%%%%%%%%%%%%%%%%%%%%%%%
\section{Classification of Finite Subgroups}
%%%%%%%%%%%%%%%%%%%%%%%%%%%%%%%%%%%%%%%%%%%%%%%%%%%%%

\begin{lemma}[Finite subgroups of SO(2) are cyclic]
    \label{lem:finite_SO2_cyclic}
    \lean{finite_subgroup_SO2_isCyclic}
    \uses{def:special_orthogonal, lem:SO2_rotations}
    \notready
    Every finite subgroup of $\SO(2)$ is cyclic.
\end{lemma}

\begin{proof}
    \notready
    $\SO(2) \cong S^1 \cong \R/\Z$. A finite subgroup of $\R/\Z$ is of the form $\frac{1}{n}\Z/\Z \cong \Z/n\Z$.
\end{proof}

\begin{theorem}[Classification of finite subgroups of O(2)]
    \label{thm:finite_O2_classification}
    \lean{finite_subgroup_O2_classification}
    \uses{def:cyclic_point_group, def:dihedral_point_group, lem:finite_SO2_cyclic, lem:O2_structure}
    \notready
    Every finite subgroup of $\Orth(2)$ is isomorphic to either $\Cyc{n}$ or $\Dih{n}$ for some $n \geq 1$.
\end{theorem}

\begin{proof}
    \uses{lem:dihedral_structure}
    \notready
    Let $H \subset \Orth(2)$ be finite. Set $H^+ = H \cap \SO(2)$.

    \textbf{Case 1}: $H \subset \SO(2)$. By Lemma~\ref{lem:finite_SO2_cyclic}, $H \cong \Cyc{n}$.

    \textbf{Case 2}: $H \not\subset \SO(2)$. Then $H^+$ has index 2 in $H$. By Lemma~\ref{lem:finite_SO2_cyclic}, $H^+ \cong \Cyc{n}$. Since $H$ contains a reflection $S$ and $S H^+ S^{-1} = H^+$, we get $H \cong \Dih{n}$.
\end{proof}

%%%%%%%%%%%%%%%%%%%%%%%%%%%%%%%%%%%%%%%%%%%%%%%%%%%%%
\section{Special Cases}
%%%%%%%%%%%%%%%%%%%%%%%%%%%%%%%%%%%%%%%%%%%%%%%%%%%%%

\begin{lemma}[C1 is trivial]
    \label{lem:C1_trivial}
    \lean{CyclicPointGroup.one}
    \uses{def:cyclic_point_group}
    \notready
    $\Cyc{1} = \{I\}$ is the trivial group.
\end{lemma}

\begin{lemma}[C2 is generated by 180-degree rotation]
    \label{lem:C2_rotation}
    \lean{CyclicPointGroup.two}
    \uses{def:cyclic_point_group}
    \notready
    $\Cyc{2} = \{I, R_\pi\}$ where $R_\pi = -I$ is the 180-degree rotation.
\end{lemma}

\begin{lemma}[D1 is a single reflection]
    \label{lem:D1_reflection}
    \lean{DihedralPointGroup.one}
    \uses{def:dihedral_point_group}
    \notready
    $\Dih{1} = \{I, S_0\} \cong \Cyc{2}$.
\end{lemma}

\begin{lemma}[D2 is the Klein four-group]
    \label{lem:D2_klein}
    \lean{DihedralPointGroup.two}
    \uses{def:dihedral_point_group}
    \notready
    $\Dih{2} = \{I, R_\pi, S_0, S_\pi\}$ is the Klein four-group (two perpendicular reflection axes and 180-degree rotation).
\end{lemma}

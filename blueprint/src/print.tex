\documentclass[a4paper, 11pt]{report}

\usepackage{amsmath, amssymb, amsthm}
\usepackage{hyperref}
\usepackage{geometry}
\geometry{margin=1in}

% Load common macros
% Common macros for the Wallpaper Groups blueprint

% Standard math packages are loaded by the document class

% Custom commands
\newcommand{\R}{\mathbb{R}}
\newcommand{\Z}{\mathbb{Z}}
\newcommand{\C}{\mathbb{C}}
\newcommand{\N}{\mathbb{N}}

% Groups
\newcommand{\Orth}{\mathrm{O}}
\newcommand{\SO}{\mathrm{SO}}
\newcommand{\Euc}{\mathrm{E}}
\newcommand{\GL}{\mathrm{GL}}
\newcommand{\Aut}{\mathrm{Aut}}

% Cyclic and dihedral groups
\newcommand{\Cyc}[1]{C_{#1}}
\newcommand{\Dih}[1]{D_{#1}}

% Semidirect product
\newcommand{\sdp}{\rtimes}

% Lattice
\newcommand{\Lat}{\Lambda}

% Trace
\newcommand{\tr}{\mathrm{tr}}

% Point group and translation subgroup
\newcommand{\PG}{\mathrm{P}}
\newcommand{\TG}{\mathrm{T}}


% Theorem environments
\newtheorem{theorem}{Theorem}[chapter]
\newtheorem{lemma}[theorem]{Lemma}
\newtheorem{proposition}[theorem]{Proposition}
\newtheorem{corollary}[theorem]{Corollary}

\theoremstyle{definition}
\newtheorem{definition}[theorem]{Definition}
\newtheorem{example}[theorem]{Example}

\theoremstyle{remark}
\newtheorem{remark}[theorem]{Remark}

% Stub out blueprint-specific commands for print version
\newcommand{\lean}[1]{}
\newcommand{\leanok}{}
\newcommand{\notready}{}
\newcommand{\mathlibok}{}
\newcommand{\uses}[1]{}
\newcommand{\proves}[1]{}
\newcommand{\discussion}[1]{}

\title{Classification of the 17 Wallpaper Groups\\[1em]
\large A Lean 4 Formalization Blueprint}
\author{Eric}
\date{\today}

\begin{document}

\maketitle

\begin{abstract}
This blueprint describes the formalization of the classification of wallpaper groups (plane crystallographic groups) in Lean 4 using Mathlib. The main result is that there are exactly 17 wallpaper groups up to isomorphism. A wallpaper group is defined as a discrete cocompact subgroup of the Euclidean group $\Euc(2)$.

The proof proceeds by:
\begin{enumerate}
    \item Establishing the crystallographic restriction theorem (rotation orders in $\{1,2,3,4,6\}$)
    \item Classifying the 10 crystallographic point groups
    \item Classifying the 5 Bravais lattice types
    \item Enumerating all compatible (lattice, point group) extensions
\end{enumerate}
\end{abstract}

\tableofcontents

% Main content file - imports all chapters

\chapter{Introduction}

\section{Project Overview}

The goal of this project is to formalize in Lean 4 + Mathlib4 the classification theorem for wallpaper groups: there are exactly 17 wallpaper groups, up to isomorphism.

A \emph{wallpaper group} (also called a \emph{plane crystallographic group}) is a discrete cocompact subgroup of the Euclidean group $\Euc(2)$, the group of isometries of the Euclidean plane. These groups describe the symmetries of repeating patterns in the plane, such as those found in wallpaper, tilings, and crystal structures.

\section{Main Results}

The project culminates in two main theorems:

\begin{enumerate}
    \item \textbf{Classification}: Every wallpaper group is isomorphic to exactly one of 17 specific groups, denoted using IUCr notation: p1, p2, pm, pg, cm, pmm, pmg, pgg, cmm, p4, p4m, p4g, p3, p3m1, p31m, p6, p6m.
    \item \textbf{Completeness}: There are no other wallpaper groups.
\end{enumerate}

\section{Proof Strategy}

Our approach follows the classical classification:

\begin{enumerate}
    \item Define the Euclidean group $\Euc(2) = \R^2 \sdp \Orth(2)$ as a semidirect product.
    \item Prove the \textbf{crystallographic restriction theorem}: rotations preserving a lattice can only have order 1, 2, 3, 4, or 6.
    \item Classify the \textbf{crystallographic point groups}: the 10 finite subgroups of $\Orth(2)$ satisfying the crystallographic restriction.
    \item Classify the \textbf{5 Bravais lattice types} in 2D.
    \item For each compatible (lattice, point group) pair, enumerate all possible group extensions.
    \item Prove that this enumeration yields exactly 17 non-isomorphic groups.
\end{enumerate}

\section{Mathlib4 Dependencies}

This project builds on the following Mathlib4 infrastructure:
\begin{itemize}
    \item \texttt{EuclideanSpace $\R$ (Fin 2)} --- the Euclidean plane
    \item \texttt{LinearIsometryEquiv} --- orthogonal group elements
    \item \texttt{SemidirectProduct} --- semidirect product construction
    \item \texttt{IsZLattice} --- $\Z$-lattice structure
    \item \texttt{DihedralGroup n} --- abstract dihedral groups
\end{itemize}

\chapter{The Euclidean Group}

This chapter defines the Euclidean group $\Euc(2)$ and establishes its basic properties. The Euclidean group is the group of all distance-preserving transformations (isometries) of the plane.

%%%%%%%%%%%%%%%%%%%%%%%%%%%%%%%%%%%%%%%%%%%%%%%%%%%%%
\section{The Euclidean Plane}
%%%%%%%%%%%%%%%%%%%%%%%%%%%%%%%%%%%%%%%%%%%%%%%%%%%%%

\begin{definition}[Euclidean plane]
    \label{def:euclidean_plane}
    \lean{EuclideanPlane}
    \leanok
    The \emph{Euclidean plane} is the vector space $\R^2$ equipped with the standard inner product $\langle (x_1, y_1), (x_2, y_2) \rangle = x_1 x_2 + y_1 y_2$.

    In Mathlib, this is \texttt{EuclideanSpace $\R$ (Fin 2)}.
\end{definition}

%%%%%%%%%%%%%%%%%%%%%%%%%%%%%%%%%%%%%%%%%%%%%%%%%%%%%
\section{The Orthogonal Group}
%%%%%%%%%%%%%%%%%%%%%%%%%%%%%%%%%%%%%%%%%%%%%%%%%%%%%

\begin{definition}[Orthogonal group]
    \label{def:orthogonal_group}
    \lean{OrthogonalGroup2}
    \uses{def:euclidean_plane}
    \notready
    The \emph{orthogonal group} $\Orth(2)$ is the group of linear isometries of $\R^2$:
    \[
        \Orth(2) = \{ A \in \GL_2(\R) \mid A^T A = I \}.
    \]

    In Mathlib, elements are \texttt{EuclideanPlane $\simeq^{li}_\R$ EuclideanPlane} (linear isometry equivalences).
\end{definition}

\begin{definition}[Special orthogonal group]
    \label{def:special_orthogonal}
    \lean{SpecialOrthogonalGroup2}
    \uses{def:orthogonal_group}
    \notready
    The \emph{special orthogonal group} $\SO(2)$ is the subgroup of $\Orth(2)$ consisting of matrices with determinant $+1$:
    \[
        \SO(2) = \{ A \in \Orth(2) \mid \det(A) = 1 \}.
    \]
    These are the \emph{rotations}.
\end{definition}

\begin{definition}[Rotation matrix]
    \label{def:rotation_matrix}
    \lean{rotationMatrix}
    \uses{def:special_orthogonal}
    \notready
    For $\theta \in \R$, the \emph{rotation matrix} by angle $\theta$ is:
    \[
        R_\theta = \begin{pmatrix} \cos\theta & -\sin\theta \\ \sin\theta & \cos\theta \end{pmatrix}.
    \]
\end{definition}

\begin{definition}[Reflection matrix]
    \label{def:reflection_matrix}
    \lean{reflectionMatrix}
    \uses{def:orthogonal_group}
    \notready
    The \emph{reflection} across the line at angle $\theta/2$ from the $x$-axis is:
    \[
        S_\theta = \begin{pmatrix} \cos\theta & \sin\theta \\ \sin\theta & -\cos\theta \end{pmatrix}.
    \]
\end{definition}

\begin{lemma}[Rotation is in SO(2)]
    \label{lem:rotation_in_SO2}
    \lean{rotationMatrix_mem_SO2}
    \uses{def:rotation_matrix, def:special_orthogonal}
    \notready
    For all $\theta$, $R_\theta \in \SO(2)$.
\end{lemma}

\begin{proof}
    \notready
    Direct computation: $R_\theta^T R_\theta = I$ and $\det(R_\theta) = \cos^2\theta + \sin^2\theta = 1$.
\end{proof}

\begin{lemma}[Reflection has determinant -1]
    \label{lem:reflection_det}
    \lean{reflectionMatrix_det}
    \uses{def:reflection_matrix}
    \notready
    For all $\theta$, $\det(S_\theta) = -1$.
\end{lemma}

\begin{proof}
    \notready
    $\det(S_\theta) = -\cos^2\theta - \sin^2\theta = -1$.
\end{proof}

\begin{lemma}[SO(2) consists of rotations]
    \label{lem:SO2_rotations}
    \lean{SO2_eq_rotations}
    \uses{def:special_orthogonal, def:rotation_matrix}
    \notready
    Every element of $\SO(2)$ is of the form $R_\theta$ for some $\theta \in [0, 2\pi)$.
\end{lemma}

\begin{proof}
    \uses{lem:rotation_in_SO2}
    \notready
    If $A = \begin{pmatrix} a & b \\ c & d \end{pmatrix} \in \SO(2)$, then $A^T A = I$ gives $a^2 + c^2 = 1$, so $(a, c) = (\cos\theta, \sin\theta)$ for some $\theta$. Orthonormality of columns and $\det = 1$ force $(b, d) = (-\sin\theta, \cos\theta)$.
\end{proof}

\begin{lemma}[O(2) is rotations and reflections]
    \label{lem:O2_structure}
    \lean{O2_eq_rotations_union_reflections}
    \uses{def:orthogonal_group, lem:SO2_rotations, def:reflection_matrix}
    \notready
    Every element of $\Orth(2)$ is either a rotation $R_\theta$ or a reflection $S_\theta$.
\end{lemma}

\begin{proof}
    \uses{lem:reflection_det}
    \notready
    If $\det(A) = 1$, use Lemma~\ref{lem:SO2_rotations}. If $\det(A) = -1$, write $A = S_0 \cdot B$ where $B = S_0 A$ has $\det(B) = 1$, so $B = R_\phi$ and $A = S_0 R_\phi = S_\phi$.
\end{proof}

%%%%%%%%%%%%%%%%%%%%%%%%%%%%%%%%%%%%%%%%%%%%%%%%%%%%%
\section{The Euclidean Group as Semidirect Product}
%%%%%%%%%%%%%%%%%%%%%%%%%%%%%%%%%%%%%%%%%%%%%%%%%%%%%

\begin{definition}[Euclidean group]
    \label{def:euclidean_group}
    \lean{EuclideanGroup2}
    \uses{def:euclidean_plane, def:orthogonal_group}
    \notready
    The \emph{Euclidean group} $\Euc(2)$ is the semidirect product
    \[
        \Euc(2) = \R^2 \sdp \Orth(2)
    \]
    where $\Orth(2)$ acts on $\R^2$ by matrix multiplication. An element $(v, A) \in \Euc(2)$ represents the affine isometry $x \mapsto Ax + v$.
\end{definition}

\begin{lemma}[Euclidean group multiplication]
    \label{lem:euclidean_mul}
    \lean{EuclideanGroup2.mul_def}
    \uses{def:euclidean_group}
    \notready
    In $\Euc(2)$:
    \[
        (v_1, A_1) \cdot (v_2, A_2) = (v_1 + A_1 v_2, A_1 A_2).
    \]
\end{lemma}

\begin{proof}
    \notready
    Follows from the semidirect product definition.
\end{proof}

\begin{lemma}[Euclidean group inverse]
    \label{lem:euclidean_inv}
    \lean{EuclideanGroup2.inv_def}
    \uses{def:euclidean_group}
    \notready
    In $\Euc(2)$:
    \[
        (v, A)^{-1} = (-A^{-1}v, A^{-1}).
    \]
\end{lemma}

\begin{proof}
    \uses{lem:euclidean_mul}
    \notready
    Verify $(v, A) \cdot (-A^{-1}v, A^{-1}) = (v - A A^{-1} v, I) = (0, I)$.
\end{proof}

\begin{definition}[Translation subgroup]
    \label{def:translation_subgroup}
    \lean{translationSubgroup}
    \uses{def:euclidean_group}
    \notready
    The \emph{translation subgroup} $T \subset \Euc(2)$ consists of elements $(v, I)$ for $v \in \R^2$.
\end{definition}

\begin{lemma}[Translation subgroup is normal]
    \label{lem:translation_normal}
    \lean{translationSubgroup_normal}
    \uses{def:translation_subgroup}
    \notready
    $T$ is a normal subgroup of $\Euc(2)$.
\end{lemma}

\begin{proof}
    \uses{lem:euclidean_mul, lem:euclidean_inv}
    \notready
    For $(v, A) \in \Euc(2)$ and $(w, I) \in T$:
    $(v, A)(w, I)(v, A)^{-1} = (v + Aw, A)(-A^{-1}v, A^{-1}) = (Aw, I) \in T$.
\end{proof}

\begin{lemma}[Quotient by translations]
    \label{lem:euclidean_quotient}
    \lean{EuclideanGroup2.quotient_translations}
    \uses{def:translation_subgroup, lem:translation_normal, def:orthogonal_group}
    \notready
    $\Euc(2) / T \cong \Orth(2)$.
\end{lemma}

\begin{proof}
    \notready
    The projection $(v, A) \mapsto A$ is a surjective homomorphism with kernel $T$.
\end{proof}

\begin{definition}[Glide reflection]
    \label{def:glide_reflection}
    \lean{glideReflection}
    \uses{def:euclidean_group, def:reflection_matrix}
    \notready
    A \emph{glide reflection} is an element $(v, S_\theta) \in \Euc(2)$ where $v$ is parallel to the reflection axis (i.e., $v$ is an eigenvector of $S_\theta$ with eigenvalue $+1$), and $v \neq 0$.
\end{definition}

\begin{lemma}[Glide reflection squared is translation]
    \label{lem:glide_squared}
    \lean{glideReflection_sq}
    \uses{def:glide_reflection}
    \notready
    If $(v, S)$ is a glide reflection, then $(v, S)^2 = (2v, I)$ is a translation by $2v$.
\end{lemma}

\begin{proof}
    \uses{lem:euclidean_mul}
    \notready
    $(v, S)^2 = (v + Sv, S^2) = (v + v, I) = (2v, I)$ since $Sv = v$ (as $v$ is along the reflection axis).
\end{proof}

\chapter{Point Groups}

This chapter classifies the finite subgroups of $\Orth(2)$. These are the ``point groups'' that will appear as quotients of wallpaper groups.

%%%%%%%%%%%%%%%%%%%%%%%%%%%%%%%%%%%%%%%%%%%%%%%%%%%%%
\section{Cyclic and Dihedral Subgroups}
%%%%%%%%%%%%%%%%%%%%%%%%%%%%%%%%%%%%%%%%%%%%%%%%%%%%%

\begin{definition}[Cyclic point group]
    \label{def:cyclic_point_group}
    \lean{CyclicPointGroup}
    \uses{def:rotation_matrix}
    \notready
    For $n \geq 1$, the \emph{cyclic point group} $\Cyc{n} \subset \Orth(2)$ is the subgroup generated by the rotation $R_{2\pi/n}$:
    \[
        \Cyc{n} = \langle R_{2\pi/n} \rangle = \{ R_{2\pi k/n} \mid k = 0, 1, \ldots, n-1 \}.
    \]
    This is the group of rotational symmetries of a regular $n$-gon.
\end{definition}

\begin{lemma}[Cyclic group order]
    \label{lem:cyclic_order}
    \lean{CyclicPointGroup.card}
    \uses{def:cyclic_point_group}
    \notready
    $|\Cyc{n}| = n$.
\end{lemma}

\begin{proof}
    \notready
    $R_{2\pi/n}$ has order exactly $n$ since $(R_{2\pi/n})^k = R_{2\pi k/n} = I$ iff $n \mid k$.
\end{proof}

\begin{definition}[Dihedral point group]
    \label{def:dihedral_point_group}
    \lean{DihedralPointGroup}
    \uses{def:cyclic_point_group, def:reflection_matrix}
    \notready
    For $n \geq 1$, the \emph{dihedral point group} $\Dih{n} \subset \Orth(2)$ is generated by $R_{2\pi/n}$ and the reflection $S_0$:
    \[
        \Dih{n} = \langle R_{2\pi/n}, S_0 \rangle.
    \]
    This is the group of all symmetries (rotations and reflections) of a regular $n$-gon.
\end{definition}

\begin{lemma}[Dihedral group order]
    \label{lem:dihedral_order}
    \lean{DihedralPointGroup.card}
    \uses{def:dihedral_point_group}
    \notready
    $|\Dih{n}| = 2n$.
\end{lemma}

\begin{proof}
    \uses{lem:cyclic_order}
    \notready
    $\Dih{n}$ contains $n$ rotations (from $\Cyc{n}$) and $n$ reflections $\{S_0, S_{2\pi/n}, \ldots, S_{2\pi(n-1)/n}\}$.
\end{proof}

\begin{lemma}[Dihedral group structure]
    \label{lem:dihedral_structure}
    \lean{DihedralPointGroup.semidirectProduct}
    \uses{def:dihedral_point_group, def:cyclic_point_group}
    \notready
    $\Dih{n} \cong \Cyc{n} \sdp \Cyc{2}$ where $\Cyc{2}$ acts on $\Cyc{n}$ by inversion.
\end{lemma}

\begin{proof}
    \notready
    The map $\Cyc{n} \to \Dih{n}$ is injective with index 2. Conjugation by $S_0$ satisfies $S_0 R_\theta S_0 = R_{-\theta}$, giving the inversion action.
\end{proof}

\begin{lemma}[Dihedral embedding from Mathlib]
    \label{lem:dihedral_embedding}
    \lean{DihedralPointGroup.equivDihedralGroup}
    \uses{def:dihedral_point_group}
    \notready
    There is a group isomorphism between $\Dih{n} \subset \Orth(2)$ and \texttt{DihedralGroup n} from Mathlib.
\end{lemma}

%%%%%%%%%%%%%%%%%%%%%%%%%%%%%%%%%%%%%%%%%%%%%%%%%%%%%
\section{Classification of Finite Subgroups}
%%%%%%%%%%%%%%%%%%%%%%%%%%%%%%%%%%%%%%%%%%%%%%%%%%%%%

\begin{lemma}[Finite subgroups of SO(2) are cyclic]
    \label{lem:finite_SO2_cyclic}
    \lean{finite_subgroup_SO2_isCyclic}
    \uses{def:special_orthogonal, lem:SO2_rotations}
    \notready
    Every finite subgroup of $\SO(2)$ is cyclic.
\end{lemma}

\begin{proof}
    \notready
    $\SO(2) \cong S^1 \cong \R/\Z$. A finite subgroup of $\R/\Z$ is of the form $\frac{1}{n}\Z/\Z \cong \Z/n\Z$.
\end{proof}

\begin{theorem}[Classification of finite subgroups of O(2)]
    \label{thm:finite_O2_classification}
    \lean{finite_subgroup_O2_classification}
    \uses{def:cyclic_point_group, def:dihedral_point_group, lem:finite_SO2_cyclic, lem:O2_structure}
    \notready
    Every finite subgroup of $\Orth(2)$ is isomorphic to either $\Cyc{n}$ or $\Dih{n}$ for some $n \geq 1$.
\end{theorem}

\begin{proof}
    \uses{lem:dihedral_structure}
    \notready
    Let $H \subset \Orth(2)$ be finite. Set $H^+ = H \cap \SO(2)$.

    \textbf{Case 1}: $H \subset \SO(2)$. By Lemma~\ref{lem:finite_SO2_cyclic}, $H \cong \Cyc{n}$.

    \textbf{Case 2}: $H \not\subset \SO(2)$. Then $H^+$ has index 2 in $H$. By Lemma~\ref{lem:finite_SO2_cyclic}, $H^+ \cong \Cyc{n}$. Since $H$ contains a reflection $S$ and $S H^+ S^{-1} = H^+$, we get $H \cong \Dih{n}$.
\end{proof}

%%%%%%%%%%%%%%%%%%%%%%%%%%%%%%%%%%%%%%%%%%%%%%%%%%%%%
\section{Special Cases}
%%%%%%%%%%%%%%%%%%%%%%%%%%%%%%%%%%%%%%%%%%%%%%%%%%%%%

\begin{lemma}[C1 is trivial]
    \label{lem:C1_trivial}
    \lean{CyclicPointGroup.one}
    \uses{def:cyclic_point_group}
    \notready
    $\Cyc{1} = \{I\}$ is the trivial group.
\end{lemma}

\begin{lemma}[C2 is generated by 180-degree rotation]
    \label{lem:C2_rotation}
    \lean{CyclicPointGroup.two}
    \uses{def:cyclic_point_group}
    \notready
    $\Cyc{2} = \{I, R_\pi\}$ where $R_\pi = -I$ is the 180-degree rotation.
\end{lemma}

\begin{lemma}[D1 is a single reflection]
    \label{lem:D1_reflection}
    \lean{DihedralPointGroup.one}
    \uses{def:dihedral_point_group}
    \notready
    $\Dih{1} = \{I, S_0\} \cong \Cyc{2}$.
\end{lemma}

\begin{lemma}[D2 is the Klein four-group]
    \label{lem:D2_klein}
    \lean{DihedralPointGroup.two}
    \uses{def:dihedral_point_group}
    \notready
    $\Dih{2} = \{I, R_\pi, S_0, S_\pi\}$ is the Klein four-group (two perpendicular reflection axes and 180-degree rotation).
\end{lemma}

\chapter{Lattices and Bravais Types}

This chapter develops the theory of 2D lattices and classifies them into 5 Bravais types based on their symmetry groups.

%%%%%%%%%%%%%%%%%%%%%%%%%%%%%%%%%%%%%%%%%%%%%%%%%%%%%
\section{Lattice Basics}
%%%%%%%%%%%%%%%%%%%%%%%%%%%%%%%%%%%%%%%%%%%%%%%%%%%%%

\begin{definition}[2D Lattice]
    \label{def:lattice}
    \lean{Lattice2}
    \uses{def:euclidean_plane}
    \notready
    A \emph{(2-dimensional) lattice} $\Lat \subset \R^2$ is a discrete subgroup isomorphic to $\Z^2$. Equivalently, $\Lat = \Z a + \Z b$ for some linearly independent vectors $a, b \in \R^2$.

    In Mathlib, we use \texttt{IsZLattice} with the condition that the lattice has rank 2.
\end{definition}

\begin{definition}[Lattice basis]
    \label{def:lattice_basis}
    \lean{latticeBasis}
    \uses{def:lattice}
    \notready
    A \emph{basis} for a lattice $\Lat$ is a pair of vectors $(a, b)$ such that $\Lat = \Z a + \Z b$.
\end{definition}

\begin{lemma}[Lattice is discrete]
    \label{lem:lattice_discrete}
    \lean{Lattice2.discrete}
    \uses{def:lattice}
    \notready
    Every lattice is a discrete subgroup of $(\R^2, +)$.
\end{lemma}

\begin{proof}
    \notready
    This is part of the definition/Mathlib's \texttt{IsZLattice}.
\end{proof}

\begin{lemma}[Lattice cocompactness]
    \label{lem:lattice_cocompact}
    \lean{Lattice2.cocompact}
    \uses{def:lattice}
    \notready
    For any lattice $\Lat$, the quotient $\R^2/\Lat$ is compact (a torus).
\end{lemma}

\begin{proof}
    \uses{def:lattice_basis}
    \notready
    The fundamental domain is the parallelogram spanned by any basis, which is compact.
\end{proof}

\begin{definition}[Fundamental domain]
    \label{def:fundamental_domain}
    \lean{latticeFundamentalDomain}
    \uses{def:lattice, def:lattice_basis}
    \notready
    For a lattice $\Lat$ with basis $(a, b)$, the \emph{fundamental domain} is:
    \[
        F = \{ s a + t b \mid 0 \leq s, t < 1 \}.
    \]
\end{definition}

%%%%%%%%%%%%%%%%%%%%%%%%%%%%%%%%%%%%%%%%%%%%%%%%%%%%%
\section{Lattice Symmetry}
%%%%%%%%%%%%%%%%%%%%%%%%%%%%%%%%%%%%%%%%%%%%%%%%%%%%%

\begin{definition}[Lattice symmetry group]
    \label{def:lattice_symmetry}
    \lean{latticeSymmetryGroup}
    \uses{def:lattice, def:orthogonal_group}
    \notready
    The \emph{symmetry group} (or \emph{holohedry}) of a lattice $\Lat$ is:
    \[
        \text{Sym}(\Lat) = \{ A \in \Orth(2) \mid A(\Lat) = \Lat \}.
    \]
\end{definition}

\begin{lemma}[Symmetry group is finite]
    \label{lem:symmetry_finite}
    \lean{latticeSymmetryGroup_finite}
    \uses{def:lattice_symmetry}
    \notready
    For any lattice $\Lat$, the symmetry group $\text{Sym}(\Lat)$ is finite.
\end{lemma}

\begin{proof}
    \notready
    $A \in \text{Sym}(\Lat)$ permutes the finitely many shortest nonzero vectors in $\Lat$.
\end{proof}

\begin{definition}[Lattice preserving]
    \label{def:lattice_preserving}
    \lean{IsLatticePreserving}
    \uses{def:lattice}
    \notready
    A linear map $A : \R^2 \to \R^2$ is \emph{$\Lat$-preserving} if $A(\Lat) \subseteq \Lat$.
\end{definition}

\begin{lemma}[Preserving iff integer matrix]
    \label{lem:preserving_integer}
    \lean{isLatticePreserving_iff_integerMatrix}
    \uses{def:lattice_preserving, def:lattice_basis}
    \notready
    Let $(a, b)$ be a basis for $\Lat$. Then $A$ preserves $\Lat$ if and only if the matrix of $A$ in the basis $(a, b)$ has integer entries.
\end{lemma}

\begin{proof}
    \notready
    $A(a), A(b) \in \Lat$ iff they are integer linear combinations of $a$ and $b$.
\end{proof}

%%%%%%%%%%%%%%%%%%%%%%%%%%%%%%%%%%%%%%%%%%%%%%%%%%%%%
\section{The Five Bravais Lattice Types}
%%%%%%%%%%%%%%%%%%%%%%%%%%%%%%%%%%%%%%%%%%%%%%%%%%%%%

\begin{definition}[Oblique lattice]
    \label{def:oblique_lattice}
    \lean{ObliqueLattice}
    \uses{def:lattice, def:lattice_symmetry}
    \notready
    A lattice is \emph{oblique} if its symmetry group is $\Cyc{2}$ (the minimal symmetry: only $\pm I$).
\end{definition}

\begin{definition}[Rectangular lattice]
    \label{def:rectangular_lattice}
    \lean{RectangularLattice}
    \uses{def:lattice, def:lattice_symmetry}
    \notready
    A lattice is \emph{rectangular} if it has a basis $(a, b)$ with $a \perp b$ and $|a| \neq |b|$. Its symmetry group is $\Dih{2}$.
\end{definition}

\begin{definition}[Centered rectangular lattice]
    \label{def:centered_rectangular_lattice}
    \lean{CenteredRectangularLattice}
    \uses{def:lattice, def:lattice_symmetry}
    \notready
    A lattice is \emph{centered rectangular} (or \emph{rhombic}) if it has a basis $(a, b)$ with $|a| = |b|$ but $a \not\perp b$ and $a \neq \pm b$. Its symmetry group is $\Dih{2}$.
\end{definition}

\begin{definition}[Square lattice]
    \label{def:square_lattice}
    \lean{SquareLattice}
    \uses{def:lattice, def:lattice_symmetry}
    \notready
    A lattice is \emph{square} if it has a basis $(a, b)$ with $a \perp b$ and $|a| = |b|$. Its symmetry group is $\Dih{4}$.
\end{definition}

\begin{definition}[Hexagonal lattice]
    \label{def:hexagonal_lattice}
    \lean{HexagonalLattice}
    \uses{def:lattice, def:lattice_symmetry}
    \notready
    A lattice is \emph{hexagonal} if it has a basis $(a, b)$ with $|a| = |b|$ and angle $60°$ or $120°$ between them. Its symmetry group is $\Dih{6}$.
\end{definition}

%%%%%%%%%%%%%%%%%%%%%%%%%%%%%%%%%%%%%%%%%%%%%%%%%%%%%
\section{Bravais Classification}
%%%%%%%%%%%%%%%%%%%%%%%%%%%%%%%%%%%%%%%%%%%%%%%%%%%%%

\begin{lemma}[Oblique symmetry]
    \label{lem:oblique_symmetry}
    \lean{ObliqueLattice.symmetryGroup}
    \uses{def:oblique_lattice, def:cyclic_point_group}
    \notready
    The symmetry group of an oblique lattice is $\Cyc{2}$.
\end{lemma}

\begin{lemma}[Rectangular symmetry]
    \label{lem:rectangular_symmetry}
    \lean{RectangularLattice.symmetryGroup}
    \uses{def:rectangular_lattice, def:dihedral_point_group}
    \notready
    The symmetry group of a rectangular lattice is $\Dih{2}$.
\end{lemma}

\begin{lemma}[Centered rectangular symmetry]
    \label{lem:centered_rectangular_symmetry}
    \lean{CenteredRectangularLattice.symmetryGroup}
    \uses{def:centered_rectangular_lattice, def:dihedral_point_group}
    \notready
    The symmetry group of a centered rectangular lattice is $\Dih{2}$.
\end{lemma}

\begin{lemma}[Square symmetry]
    \label{lem:square_symmetry}
    \lean{SquareLattice.symmetryGroup}
    \uses{def:square_lattice, def:dihedral_point_group}
    \notready
    The symmetry group of a square lattice is $\Dih{4}$.
\end{lemma}

\begin{lemma}[Hexagonal symmetry]
    \label{lem:hexagonal_symmetry}
    \lean{HexagonalLattice.symmetryGroup}
    \uses{def:hexagonal_lattice, def:dihedral_point_group}
    \notready
    The symmetry group of a hexagonal lattice is $\Dih{6}$.
\end{lemma}

\begin{theorem}[Bravais classification]
    \label{thm:bravais_classification}
    \lean{bravais_classification}
    \uses{def:oblique_lattice, def:rectangular_lattice, def:centered_rectangular_lattice, def:square_lattice, def:hexagonal_lattice, lem:symmetry_finite}
    \notready
    Every 2D lattice is equivalent (under $\GL_2(\R)$) to exactly one of the five types: oblique, rectangular, centered rectangular, square, or hexagonal.
\end{theorem}

\begin{proof}
    \uses{lem:oblique_symmetry, lem:rectangular_symmetry, lem:centered_rectangular_symmetry, lem:square_symmetry, lem:hexagonal_symmetry}
    \notready
    Classify by the symmetry group, which must be a finite subgroup of $\Orth(2)$:
    \begin{itemize}
        \item $\Cyc{2}$: oblique
        \item $\Dih{2}$: rectangular or centered rectangular (distinguished by whether shortest vectors are perpendicular)
        \item $\Dih{4}$: square
        \item $\Dih{6}$: hexagonal
    \end{itemize}
    Other finite subgroups of $\Orth(2)$ cannot occur as lattice symmetry groups (by the crystallographic restriction).
\end{proof}

\chapter{Crystallographic Restriction}

This chapter proves the crystallographic restriction theorem, which limits the possible rotation orders in lattice-preserving transformations.

%%%%%%%%%%%%%%%%%%%%%%%%%%%%%%%%%%%%%%%%%%%%%%%%%%%%%
\section{The Trace Argument}
%%%%%%%%%%%%%%%%%%%%%%%%%%%%%%%%%%%%%%%%%%%%%%%%%%%%%

\begin{lemma}[Rotation trace formula]
    \label{lem:rotation_trace}
    \lean{rotationMatrix_trace}
    \uses{def:rotation_matrix}
    \notready
    The trace of the rotation matrix $R_\theta$ is:
    \[
        \tr(R_\theta) = 2\cos\theta.
    \]
\end{lemma}

\begin{proof}
    \notready
    $\tr(R_\theta) = \cos\theta + \cos\theta = 2\cos\theta$.
\end{proof}

\begin{lemma}[Trace is basis-independent]
    \label{lem:trace_invariant}
    \lean{Matrix.trace_conj}
    \leanok
    For any invertible matrix $P$, $\tr(PAP^{-1}) = \tr(A)$.

    This is already in Mathlib.
\end{lemma}

\begin{lemma}[Integer matrix has integer trace]
    \label{lem:integer_matrix_trace}
    \lean{Matrix.trace_intCast}
    \uses{lem:preserving_integer}
    \notready
    If $M \in \text{Mat}_{2 \times 2}(\Z)$, then $\tr(M) \in \Z$.
\end{lemma}

\begin{proof}
    \notready
    The trace is the sum of diagonal entries.
\end{proof}

\begin{theorem}[Crystallographic restriction theorem]
    \label{thm:crystallographic_restriction}
    \lean{crystallographic_restriction}
    \uses{def:lattice, def:lattice_preserving, lem:rotation_trace, lem:trace_invariant, lem:integer_matrix_trace, lem:preserving_integer}
    \notready
    If a rotation $R_\theta$ preserves a lattice $\Lat$, then $\theta$ is a multiple of $60°$. Equivalently, the order of $R_\theta$ is in $\{1, 2, 3, 4, 6\}$.
\end{theorem}

\begin{proof}
    \notready
    Since $R_\theta$ preserves $\Lat$, its matrix in the lattice basis has integer entries (Lemma~\ref{lem:preserving_integer}). Therefore:
    \[
        \tr(R_\theta) = 2\cos\theta \in \Z.
    \]
    Since $|\cos\theta| \leq 1$, we have $2\cos\theta \in \{-2, -1, 0, 1, 2\}$, so:
    \[
        \cos\theta \in \{-1, -\tfrac{1}{2}, 0, \tfrac{1}{2}, 1\}.
    \]
    This gives $\theta \in \{0°, 60°, 90°, 120°, 180°\}$ (modulo $360°$), corresponding to rotation orders $\{1, 6, 4, 3, 2\}$.
\end{proof}

%%%%%%%%%%%%%%%%%%%%%%%%%%%%%%%%%%%%%%%%%%%%%%%%%%%%%
\section{Crystallographic Point Groups}
%%%%%%%%%%%%%%%%%%%%%%%%%%%%%%%%%%%%%%%%%%%%%%%%%%%%%

\begin{definition}[Crystallographic point group]
    \label{def:crystallographic_point_group}
    \lean{IsCrystallographicPointGroup}
    \uses{def:orthogonal_group, thm:crystallographic_restriction}
    \notready
    A finite subgroup $H \subset \Orth(2)$ is \emph{crystallographic} if every rotation in $H$ has order in $\{1, 2, 3, 4, 6\}$.
\end{definition}

\begin{corollary}[Enumeration of crystallographic point groups]
    \label{cor:crystallographic_enumeration}
    \lean{crystallographic_point_groups}
    \uses{def:crystallographic_point_group, thm:finite_O2_classification, thm:crystallographic_restriction}
    \notready
    The crystallographic point groups in 2D are exactly:
    \[
        \Cyc{1}, \Cyc{2}, \Cyc{3}, \Cyc{4}, \Cyc{6}, \Dih{1}, \Dih{2}, \Dih{3}, \Dih{4}, \Dih{6}.
    \]
    There are 10 such groups.
\end{corollary}

\begin{proof}
    \uses{lem:cyclic_order, lem:dihedral_order}
    \notready
    By Theorem~\ref{thm:finite_O2_classification}, every finite subgroup of $\Orth(2)$ is $\Cyc{n}$ or $\Dih{n}$. By the crystallographic restriction, only $n \in \{1, 2, 3, 4, 6\}$ are allowed.
\end{proof}

%%%%%%%%%%%%%%%%%%%%%%%%%%%%%%%%%%%%%%%%%%%%%%%%%%%%%
\section{Point Group--Lattice Compatibility}
%%%%%%%%%%%%%%%%%%%%%%%%%%%%%%%%%%%%%%%%%%%%%%%%%%%%%

\begin{lemma}[Compatible point groups]
    \label{lem:compatible_point_groups}
    \lean{compatible_point_groups}
    \uses{cor:crystallographic_enumeration, thm:bravais_classification, def:lattice_symmetry}
    \notready
    A crystallographic point group $H$ can act on a lattice $\Lat$ (i.e., $H \subseteq \text{Sym}(\Lat)$) if and only if:
    \begin{center}
    \begin{tabular}{l|l}
        \textbf{Lattice type} & \textbf{Compatible point groups} \\ \hline
        Oblique & $\Cyc{1}, \Cyc{2}$ \\
        Rectangular & $\Cyc{1}, \Cyc{2}, \Dih{1}, \Dih{2}$ \\
        Centered rectangular & $\Cyc{1}, \Cyc{2}, \Dih{1}, \Dih{2}$ \\
        Square & $\Cyc{1}, \Cyc{2}, \Cyc{4}, \Dih{1}, \Dih{2}, \Dih{4}$ \\
        Hexagonal & $\Cyc{1}, \Cyc{2}, \Cyc{3}, \Cyc{6}, \Dih{1}, \Dih{2}, \Dih{3}, \Dih{6}$
    \end{tabular}
    \end{center}
\end{lemma}

\begin{proof}
    \uses{lem:oblique_symmetry, lem:rectangular_symmetry, lem:centered_rectangular_symmetry, lem:square_symmetry, lem:hexagonal_symmetry}
    \notready
    $H$ is compatible with $\Lat$ iff $H$ is a subgroup of $\text{Sym}(\Lat)$. The full symmetry groups are given by the Bravais classification.
\end{proof}

\chapter{Wallpaper Groups: Definition and Structure}

This chapter defines wallpaper groups and establishes their basic structural properties.

%%%%%%%%%%%%%%%%%%%%%%%%%%%%%%%%%%%%%%%%%%%%%%%%%%%%%
\section{Definition}
%%%%%%%%%%%%%%%%%%%%%%%%%%%%%%%%%%%%%%%%%%%%%%%%%%%%%

\begin{definition}[Discrete subgroup]
    \label{def:discrete_subgroup}
    \lean{IsDiscreteSubgroup}
    \uses{def:euclidean_group}
    \notready
    A subgroup $\Gamma \subset \Euc(2)$ is \emph{discrete} if the induced topology on $\Gamma$ is discrete (equivalently, $\{e\}$ is open in $\Gamma$).
\end{definition}

\begin{definition}[Cocompact subgroup]
    \label{def:cocompact_subgroup}
    \lean{IsCocompact}
    \uses{def:euclidean_group, def:euclidean_plane}
    \notready
    A subgroup $\Gamma \subset \Euc(2)$ is \emph{cocompact} if the quotient space $\R^2/\Gamma$ is compact.
\end{definition}

\begin{definition}[Wallpaper group]
    \label{def:wallpaper_group}
    \lean{IsWallpaperGroup}
    \uses{def:discrete_subgroup, def:cocompact_subgroup}
    \notready
    A \emph{wallpaper group} is a subgroup $\Gamma \subset \Euc(2)$ that is both discrete and cocompact.
\end{definition}

%%%%%%%%%%%%%%%%%%%%%%%%%%%%%%%%%%%%%%%%%%%%%%%%%%%%%
\section{Translation Lattice}
%%%%%%%%%%%%%%%%%%%%%%%%%%%%%%%%%%%%%%%%%%%%%%%%%%%%%

\begin{definition}[Translation subgroup of wallpaper group]
    \label{def:wallpaper_translations}
    \lean{WallpaperGroup.translationSubgroup}
    \uses{def:wallpaper_group, def:translation_subgroup}
    \notready
    For a wallpaper group $\Gamma$, its \emph{translation subgroup} is:
    \[
        \TG(\Gamma) = \Gamma \cap T = \{ (v, I) \in \Gamma \}.
    \]
\end{definition}

\begin{lemma}[Translation subgroup is a lattice]
    \label{lem:translation_is_lattice}
    \lean{WallpaperGroup.translationSubgroup_isLattice}
    \uses{def:wallpaper_translations, def:lattice, def:discrete_subgroup, def:cocompact_subgroup}
    \notready
    For a wallpaper group $\Gamma$, the translation subgroup $\TG(\Gamma)$ is a rank-2 lattice.
\end{lemma}

\begin{proof}
    \uses{lem:lattice_discrete, lem:lattice_cocompact}
    \notready
    Discreteness of $\Gamma$ implies discreteness of $\TG(\Gamma)$. Cocompactness of $\Gamma$ implies $\TG(\Gamma)$ spans $\R^2$. A discrete subgroup of $\R^2$ that spans is a rank-2 lattice.
\end{proof}

\begin{lemma}[Translation subgroup is normal]
    \label{lem:wallpaper_translation_normal}
    \lean{WallpaperGroup.translationSubgroup_normal}
    \uses{def:wallpaper_translations, lem:translation_normal}
    \notready
    $\TG(\Gamma)$ is a normal subgroup of $\Gamma$.
\end{lemma}

\begin{proof}
    \notready
    Follows from $T$ being normal in $\Euc(2)$ (Lemma~\ref{lem:translation_normal}).
\end{proof}

%%%%%%%%%%%%%%%%%%%%%%%%%%%%%%%%%%%%%%%%%%%%%%%%%%%%%
\section{Point Group}
%%%%%%%%%%%%%%%%%%%%%%%%%%%%%%%%%%%%%%%%%%%%%%%%%%%%%

\begin{definition}[Point group of wallpaper group]
    \label{def:wallpaper_point_group}
    \lean{WallpaperGroup.pointGroup}
    \uses{def:wallpaper_group, lem:euclidean_quotient}
    \notready
    The \emph{point group} of a wallpaper group $\Gamma$ is:
    \[
        \PG(\Gamma) = \{ A \in \Orth(2) \mid \exists v : (v, A) \in \Gamma \}.
    \]
    Equivalently, $\PG(\Gamma)$ is the image of $\Gamma$ under the projection $\Euc(2) \to \Orth(2)$.
\end{definition}

\begin{lemma}[Point group is finite]
    \label{lem:point_group_finite}
    \lean{WallpaperGroup.pointGroup_finite}
    \uses{def:wallpaper_point_group, def:discrete_subgroup}
    \notready
    For a wallpaper group $\Gamma$, the point group $\PG(\Gamma)$ is finite.
\end{lemma}

\begin{proof}
    \uses{lem:translation_is_lattice}
    \notready
    $\PG(\Gamma) \cong \Gamma/\TG(\Gamma)$, and the translation lattice has finite index in $\Gamma$ since both are cocompact.
\end{proof}

\begin{lemma}[Point group preserves lattice]
    \label{lem:point_group_preserves}
    \lean{WallpaperGroup.pointGroup_preservesLattice}
    \uses{def:wallpaper_point_group, def:wallpaper_translations, def:lattice_preserving}
    \notready
    The point group $\PG(\Gamma)$ preserves the translation lattice $\TG(\Gamma)$.
\end{lemma}

\begin{proof}
    \uses{lem:wallpaper_translation_normal}
    \notready
    If $(v, A) \in \Gamma$ and $(w, I) \in \TG(\Gamma)$, then $(v, A)(w, I)(v, A)^{-1} = (Aw, I) \in \TG(\Gamma)$.
\end{proof}

\begin{lemma}[Point group is crystallographic]
    \label{lem:point_group_crystallographic}
    \lean{WallpaperGroup.pointGroup_isCrystallographic}
    \uses{def:wallpaper_point_group, def:crystallographic_point_group, lem:point_group_finite, lem:point_group_preserves}
    \notready
    For a wallpaper group $\Gamma$, the point group $\PG(\Gamma)$ is one of the 10 crystallographic point groups.
\end{lemma}

\begin{proof}
    \uses{thm:crystallographic_restriction, cor:crystallographic_enumeration}
    \notready
    $\PG(\Gamma)$ is finite and preserves a lattice, so by the crystallographic restriction, all rotations have order in $\{1, 2, 3, 4, 6\}$.
\end{proof}

%%%%%%%%%%%%%%%%%%%%%%%%%%%%%%%%%%%%%%%%%%%%%%%%%%%%%
\section{Short Exact Sequence}
%%%%%%%%%%%%%%%%%%%%%%%%%%%%%%%%%%%%%%%%%%%%%%%%%%%%%

\begin{lemma}[Wallpaper group short exact sequence]
    \label{lem:wallpaper_ses}
    \lean{WallpaperGroup.shortExactSequence}
    \uses{def:wallpaper_translations, def:wallpaper_point_group, lem:wallpaper_translation_normal}
    \notready
    For a wallpaper group $\Gamma$, there is a short exact sequence:
    \[
        1 \longrightarrow \TG(\Gamma) \longrightarrow \Gamma \longrightarrow \PG(\Gamma) \longrightarrow 1.
    \]
\end{lemma}

\begin{definition}[Symmorphic wallpaper group]
    \label{def:symmorphic}
    \lean{IsSymmorphic}
    \uses{lem:wallpaper_ses}
    \notready
    A wallpaper group $\Gamma$ is \emph{symmorphic} if the short exact sequence splits, i.e., if $\Gamma \cong \TG(\Gamma) \sdp \PG(\Gamma)$.
\end{definition}

\begin{lemma}[Symmorphic characterization]
    \label{lem:symmorphic_char}
    \lean{isSymmorphic_iff}
    \uses{def:symmorphic}
    \notready
    A wallpaper group is symmorphic if and only if it contains elements $(0, A)$ for every $A \in \PG(\Gamma)$.
\end{lemma}

\begin{proof}
    \notready
    The sequence splits iff there is a section $\PG(\Gamma) \to \Gamma$, i.e., iff each $A \in \PG(\Gamma)$ lifts to $(0, A) \in \Gamma$.
\end{proof}

\begin{lemma}[Non-symmorphic implies glide reflections]
    \label{lem:non_symmorphic_glide}
    \lean{nonSymmorphic_hasGlideReflection}
    \uses{def:symmorphic, def:glide_reflection}
    \notready
    If a wallpaper group $\Gamma$ is non-symmorphic, then $\Gamma$ contains glide reflections.
\end{lemma}

\begin{proof}
    \uses{lem:symmorphic_char, lem:glide_squared}
    \notready
    If $\Gamma$ is non-symmorphic, some reflection $S \in \PG(\Gamma)$ only lifts to $(v, S)$ with $v \neq 0$. Since $(v, S)^2 = (2v, I) \in \TG(\Gamma)$, we must have $v$ parallel to the reflection axis, so $(v, S)$ is a glide reflection.
\end{proof}

\chapter{The Seventeen Wallpaper Groups}

This chapter explicitly defines all 17 wallpaper groups using the IUCr (International Union of Crystallography) notation.

%%%%%%%%%%%%%%%%%%%%%%%%%%%%%%%%%%%%%%%%%%%%%%%%%%%%%
\section{Oblique Lattice Groups}
%%%%%%%%%%%%%%%%%%%%%%%%%%%%%%%%%%%%%%%%%%%%%%%%%%%%%

There are 2 wallpaper groups with an oblique lattice.

\begin{definition}[p1]
    \label{def:p1}
    \lean{WallpaperGroup.p1}
    \uses{def:wallpaper_group, def:oblique_lattice}
    \notready
    The group \textbf{p1} is the simplest wallpaper group:
    \begin{itemize}
        \item Lattice: oblique $\Lat$
        \item Point group: $\Cyc{1}$ (trivial)
        \item Elements: $\{(v, I) \mid v \in \Lat\}$
        \item Symmetry: translations only
    \end{itemize}
    This is the group of pure translations by a lattice.
\end{definition}

\begin{definition}[p2]
    \label{def:p2}
    \lean{WallpaperGroup.p2}
    \uses{def:wallpaper_group, def:oblique_lattice, def:cyclic_point_group}
    \notready
    The group \textbf{p2} has 180° rotation symmetry:
    \begin{itemize}
        \item Lattice: oblique $\Lat$
        \item Point group: $\Cyc{2}$
        \item Elements: $\{(v, I), (v, R_\pi) \mid v \in \Lat\}$
        \item Symmetry: translations + 180° rotations
    \end{itemize}
    This is the semidirect product $\Lat \sdp \Cyc{2}$.
\end{definition}

%%%%%%%%%%%%%%%%%%%%%%%%%%%%%%%%%%%%%%%%%%%%%%%%%%%%%
\section{Rectangular Lattice Groups}
%%%%%%%%%%%%%%%%%%%%%%%%%%%%%%%%%%%%%%%%%%%%%%%%%%%%%

There are 5 wallpaper groups with a rectangular lattice (primitive, not centered).

\begin{definition}[pm]
    \label{def:pm}
    \lean{WallpaperGroup.pm}
    \uses{def:wallpaper_group, def:rectangular_lattice, def:dihedral_point_group}
    \notready
    The group \textbf{pm} has reflection symmetry:
    \begin{itemize}
        \item Lattice: rectangular $\Lat = \Z a + \Z b$ with $a \perp b$
        \item Point group: $\Dih{1}$ (one reflection)
        \item Reflection axes: parallel to $a$, passing through lattice points
        \item Type: symmorphic
    \end{itemize}
\end{definition}

\begin{definition}[pg]
    \label{def:pg}
    \lean{WallpaperGroup.pg}
    \uses{def:wallpaper_group, def:rectangular_lattice, def:glide_reflection}
    \notready
    The group \textbf{pg} has glide reflection symmetry:
    \begin{itemize}
        \item Lattice: rectangular $\Lat = \Z a + \Z b$ with $a \perp b$
        \item Point group: $\Dih{1}$
        \item No pure reflections; only glide reflections with translation $\frac{1}{2}a$
        \item Type: non-symmorphic
    \end{itemize}
\end{definition}

\begin{definition}[pmm]
    \label{def:pmm}
    \lean{WallpaperGroup.pmm}
    \uses{def:wallpaper_group, def:rectangular_lattice, def:dihedral_point_group}
    \notready
    The group \textbf{pmm} has two perpendicular reflection axes:
    \begin{itemize}
        \item Lattice: rectangular $\Lat$
        \item Point group: $\Dih{2}$
        \item Reflection axes: parallel to both basis vectors, through lattice points
        \item 180° rotation centers at lattice points and edge midpoints
        \item Type: symmorphic
    \end{itemize}
\end{definition}

\begin{definition}[pmg]
    \label{def:pmg}
    \lean{WallpaperGroup.pmg}
    \uses{def:wallpaper_group, def:rectangular_lattice, def:glide_reflection}
    \notready
    The group \textbf{pmg} mixes reflections and glide reflections:
    \begin{itemize}
        \item Lattice: rectangular $\Lat$
        \item Point group: $\Dih{2}$
        \item One direction: pure reflections
        \item Perpendicular direction: glide reflections
        \item Type: non-symmorphic
    \end{itemize}
\end{definition}

\begin{definition}[pgg]
    \label{def:pgg}
    \lean{WallpaperGroup.pgg}
    \uses{def:wallpaper_group, def:rectangular_lattice, def:glide_reflection}
    \notready
    The group \textbf{pgg} has only glide reflections:
    \begin{itemize}
        \item Lattice: rectangular $\Lat$
        \item Point group: $\Dih{2}$
        \item Both reflection directions: glide reflections only
        \item 180° rotations at cell centers
        \item Type: non-symmorphic
    \end{itemize}
\end{definition}

%%%%%%%%%%%%%%%%%%%%%%%%%%%%%%%%%%%%%%%%%%%%%%%%%%%%%
\section{Centered Rectangular Lattice Groups}
%%%%%%%%%%%%%%%%%%%%%%%%%%%%%%%%%%%%%%%%%%%%%%%%%%%%%

There are 2 wallpaper groups with a centered rectangular (rhombic) lattice.

\begin{definition}[cm]
    \label{def:cm}
    \lean{WallpaperGroup.cm}
    \uses{def:wallpaper_group, def:centered_rectangular_lattice, def:dihedral_point_group}
    \notready
    The group \textbf{cm} has reflection symmetry with centered lattice:
    \begin{itemize}
        \item Lattice: centered rectangular $\Lat$
        \item Point group: $\Dih{1}$
        \item Reflection axes at two heights (lattice points and centers)
        \item Contains both pure reflections and glide reflections
        \item Type: symmorphic (but with glide reflections arising from centering)
    \end{itemize}
\end{definition}

\begin{definition}[cmm]
    \label{def:cmm}
    \lean{WallpaperGroup.cmm}
    \uses{def:wallpaper_group, def:centered_rectangular_lattice, def:dihedral_point_group}
    \notready
    The group \textbf{cmm} is the full symmetry of a rhombus:
    \begin{itemize}
        \item Lattice: centered rectangular $\Lat$
        \item Point group: $\Dih{2}$
        \item Two perpendicular reflection axes
        \item 180° rotations
        \item Type: symmorphic
    \end{itemize}
\end{definition}

%%%%%%%%%%%%%%%%%%%%%%%%%%%%%%%%%%%%%%%%%%%%%%%%%%%%%
\section{Square Lattice Groups}
%%%%%%%%%%%%%%%%%%%%%%%%%%%%%%%%%%%%%%%%%%%%%%%%%%%%%

There are 3 wallpaper groups with a square lattice.

\begin{definition}[p4]
    \label{def:p4}
    \lean{WallpaperGroup.p4}
    \uses{def:wallpaper_group, def:square_lattice, def:cyclic_point_group}
    \notready
    The group \textbf{p4} has 4-fold rotational symmetry:
    \begin{itemize}
        \item Lattice: square $\Lat$
        \item Point group: $\Cyc{4}$
        \item 90° rotations at lattice points
        \item 180° rotations at edge midpoints and cell centers
        \item No reflections
        \item Type: symmorphic
    \end{itemize}
\end{definition}

\begin{definition}[p4m]
    \label{def:p4m}
    \lean{WallpaperGroup.p4m}
    \uses{def:wallpaper_group, def:square_lattice, def:dihedral_point_group}
    \notready
    The group \textbf{p4m} is the full symmetry of a square lattice:
    \begin{itemize}
        \item Lattice: square $\Lat$
        \item Point group: $\Dih{4}$
        \item Reflection axes: parallel to edges and diagonals
        \item 90° and 180° rotations
        \item Type: symmorphic
    \end{itemize}
\end{definition}

\begin{definition}[p4g]
    \label{def:p4g}
    \lean{WallpaperGroup.p4g}
    \uses{def:wallpaper_group, def:square_lattice, def:glide_reflection}
    \notready
    The group \textbf{p4g} has 4-fold rotation with glide reflections:
    \begin{itemize}
        \item Lattice: square $\Lat$
        \item Point group: $\Dih{4}$
        \item 90° rotations at cell centers (not at lattice points)
        \item Reflections along diagonals
        \item Glide reflections along edges
        \item Type: non-symmorphic
    \end{itemize}
\end{definition}

%%%%%%%%%%%%%%%%%%%%%%%%%%%%%%%%%%%%%%%%%%%%%%%%%%%%%
\section{Hexagonal Lattice Groups}
%%%%%%%%%%%%%%%%%%%%%%%%%%%%%%%%%%%%%%%%%%%%%%%%%%%%%

There are 5 wallpaper groups with a hexagonal lattice.

\begin{definition}[p3]
    \label{def:p3}
    \lean{WallpaperGroup.p3}
    \uses{def:wallpaper_group, def:hexagonal_lattice, def:cyclic_point_group}
    \notready
    The group \textbf{p3} has 3-fold rotational symmetry:
    \begin{itemize}
        \item Lattice: hexagonal $\Lat$
        \item Point group: $\Cyc{3}$
        \item 120° rotations at lattice points and two other positions per cell
        \item No reflections
        \item Type: symmorphic
    \end{itemize}
\end{definition}

\begin{definition}[p3m1]
    \label{def:p3m1}
    \lean{WallpaperGroup.p3m1}
    \uses{def:wallpaper_group, def:hexagonal_lattice, def:dihedral_point_group}
    \notready
    The group \textbf{p3m1} has 3-fold rotation and reflections:
    \begin{itemize}
        \item Lattice: hexagonal $\Lat$
        \item Point group: $\Dih{3}$
        \item 120° rotations at lattice points
        \item Reflection axes pass through lattice points
        \item Type: symmorphic
    \end{itemize}
\end{definition}

\begin{definition}[p31m]
    \label{def:p31m}
    \lean{WallpaperGroup.p31m}
    \uses{def:wallpaper_group, def:hexagonal_lattice, def:dihedral_point_group}
    \notready
    The group \textbf{p31m} has 3-fold rotation and differently-oriented reflections:
    \begin{itemize}
        \item Lattice: hexagonal $\Lat$
        \item Point group: $\Dih{3}$
        \item 120° rotations at lattice points and cell centers
        \item Reflection axes pass between lattice points
        \item Distinguished from p3m1 by reflection axis orientation
        \item Type: symmorphic
    \end{itemize}
\end{definition}

\begin{definition}[p6]
    \label{def:p6}
    \lean{WallpaperGroup.p6}
    \uses{def:wallpaper_group, def:hexagonal_lattice, def:cyclic_point_group}
    \notready
    The group \textbf{p6} has 6-fold rotational symmetry:
    \begin{itemize}
        \item Lattice: hexagonal $\Lat$
        \item Point group: $\Cyc{6}$
        \item 60° rotations at lattice points
        \item 120° rotations at two other positions per cell
        \item 180° rotations at edge midpoints
        \item No reflections
        \item Type: symmorphic
    \end{itemize}
\end{definition}

\begin{definition}[p6m]
    \label{def:p6m}
    \lean{WallpaperGroup.p6m}
    \uses{def:wallpaper_group, def:hexagonal_lattice, def:dihedral_point_group}
    \notready
    The group \textbf{p6m} is the full symmetry of a hexagonal lattice:
    \begin{itemize}
        \item Lattice: hexagonal $\Lat$
        \item Point group: $\Dih{6}$
        \item 60° rotations at lattice points
        \item 6 reflection axes through each lattice point
        \item Highest symmetry of any wallpaper group (order 12 point group)
        \item Type: symmorphic
    \end{itemize}
\end{definition}

%%%%%%%%%%%%%%%%%%%%%%%%%%%%%%%%%%%%%%%%%%%%%%%%%%%%%
\section{Summary Table}
%%%%%%%%%%%%%%%%%%%%%%%%%%%%%%%%%%%%%%%%%%%%%%%%%%%%%

\begin{center}
\begin{tabular}{|l|l|l|l|l|}
\hline
\textbf{Group} & \textbf{Lattice} & \textbf{Point Group} & \textbf{Order} & \textbf{Type} \\ \hline
p1 & Oblique & $\Cyc{1}$ & 1 & Symmorphic \\
p2 & Oblique & $\Cyc{2}$ & 2 & Symmorphic \\ \hline
pm & Rectangular & $\Dih{1}$ & 2 & Symmorphic \\
pg & Rectangular & $\Dih{1}$ & 2 & Non-symmorphic \\
pmm & Rectangular & $\Dih{2}$ & 4 & Symmorphic \\
pmg & Rectangular & $\Dih{2}$ & 4 & Non-symmorphic \\
pgg & Rectangular & $\Dih{2}$ & 4 & Non-symmorphic \\ \hline
cm & Centered rect. & $\Dih{1}$ & 2 & Symmorphic \\
cmm & Centered rect. & $\Dih{2}$ & 4 & Symmorphic \\ \hline
p4 & Square & $\Cyc{4}$ & 4 & Symmorphic \\
p4m & Square & $\Dih{4}$ & 8 & Symmorphic \\
p4g & Square & $\Dih{4}$ & 8 & Non-symmorphic \\ \hline
p3 & Hexagonal & $\Cyc{3}$ & 3 & Symmorphic \\
p3m1 & Hexagonal & $\Dih{3}$ & 6 & Symmorphic \\
p31m & Hexagonal & $\Dih{3}$ & 6 & Symmorphic \\
p6 & Hexagonal & $\Cyc{6}$ & 6 & Symmorphic \\
p6m & Hexagonal & $\Dih{6}$ & 12 & Symmorphic \\ \hline
\end{tabular}
\end{center}

\chapter{Classification Theorem}

This chapter proves the main theorem: every wallpaper group is isomorphic to exactly one of the 17 groups defined in the previous chapter.

%%%%%%%%%%%%%%%%%%%%%%%%%%%%%%%%%%%%%%%%%%%%%%%%%%%%%
\section{Verification: Each Group is a Wallpaper Group}
%%%%%%%%%%%%%%%%%%%%%%%%%%%%%%%%%%%%%%%%%%%%%%%%%%%%%

\begin{lemma}[p1 is a wallpaper group]
    \label{lem:p1_is_wallpaper}
    \lean{WallpaperGroup.p1.isWallpaperGroup}
    \uses{def:p1, def:wallpaper_group}
    \notready
    The group p1 is a wallpaper group.
\end{lemma}

\begin{proof}
    \uses{lem:lattice_discrete, lem:lattice_cocompact}
    \notready
    p1 is the translation group of a lattice, which is discrete and cocompact.
\end{proof}

\begin{lemma}[p2 is a wallpaper group]
    \label{lem:p2_is_wallpaper}
    \lean{WallpaperGroup.p2.isWallpaperGroup}
    \uses{def:p2, def:wallpaper_group}
    \notready
    The group p2 is a wallpaper group.
\end{lemma}

\begin{lemma}[pm is a wallpaper group]
    \label{lem:pm_is_wallpaper}
    \lean{WallpaperGroup.pm.isWallpaperGroup}
    \uses{def:pm, def:wallpaper_group}
    \notready
    The group pm is a wallpaper group.
\end{lemma}

\begin{lemma}[pg is a wallpaper group]
    \label{lem:pg_is_wallpaper}
    \lean{WallpaperGroup.pg.isWallpaperGroup}
    \uses{def:pg, def:wallpaper_group}
    \notready
    The group pg is a wallpaper group.
\end{lemma}

\begin{lemma}[cm is a wallpaper group]
    \label{lem:cm_is_wallpaper}
    \lean{WallpaperGroup.cm.isWallpaperGroup}
    \uses{def:cm, def:wallpaper_group}
    \notready
    The group cm is a wallpaper group.
\end{lemma}

\begin{lemma}[pmm is a wallpaper group]
    \label{lem:pmm_is_wallpaper}
    \lean{WallpaperGroup.pmm.isWallpaperGroup}
    \uses{def:pmm, def:wallpaper_group}
    \notready
    The group pmm is a wallpaper group.
\end{lemma}

\begin{lemma}[pmg is a wallpaper group]
    \label{lem:pmg_is_wallpaper}
    \lean{WallpaperGroup.pmg.isWallpaperGroup}
    \uses{def:pmg, def:wallpaper_group}
    \notready
    The group pmg is a wallpaper group.
\end{lemma}

\begin{lemma}[pgg is a wallpaper group]
    \label{lem:pgg_is_wallpaper}
    \lean{WallpaperGroup.pgg.isWallpaperGroup}
    \uses{def:pgg, def:wallpaper_group}
    \notready
    The group pgg is a wallpaper group.
\end{lemma}

\begin{lemma}[cmm is a wallpaper group]
    \label{lem:cmm_is_wallpaper}
    \lean{WallpaperGroup.cmm.isWallpaperGroup}
    \uses{def:cmm, def:wallpaper_group}
    \notready
    The group cmm is a wallpaper group.
\end{lemma}

\begin{lemma}[p4 is a wallpaper group]
    \label{lem:p4_is_wallpaper}
    \lean{WallpaperGroup.p4.isWallpaperGroup}
    \uses{def:p4, def:wallpaper_group}
    \notready
    The group p4 is a wallpaper group.
\end{lemma}

\begin{lemma}[p4m is a wallpaper group]
    \label{lem:p4m_is_wallpaper}
    \lean{WallpaperGroup.p4m.isWallpaperGroup}
    \uses{def:p4m, def:wallpaper_group}
    \notready
    The group p4m is a wallpaper group.
\end{lemma}

\begin{lemma}[p4g is a wallpaper group]
    \label{lem:p4g_is_wallpaper}
    \lean{WallpaperGroup.p4g.isWallpaperGroup}
    \uses{def:p4g, def:wallpaper_group}
    \notready
    The group p4g is a wallpaper group.
\end{lemma}

\begin{lemma}[p3 is a wallpaper group]
    \label{lem:p3_is_wallpaper}
    \lean{WallpaperGroup.p3.isWallpaperGroup}
    \uses{def:p3, def:wallpaper_group}
    \notready
    The group p3 is a wallpaper group.
\end{lemma}

\begin{lemma}[p3m1 is a wallpaper group]
    \label{lem:p3m1_is_wallpaper}
    \lean{WallpaperGroup.p3m1.isWallpaperGroup}
    \uses{def:p3m1, def:wallpaper_group}
    \notready
    The group p3m1 is a wallpaper group.
\end{lemma}

\begin{lemma}[p31m is a wallpaper group]
    \label{lem:p31m_is_wallpaper}
    \lean{WallpaperGroup.p31m.isWallpaperGroup}
    \uses{def:p31m, def:wallpaper_group}
    \notready
    The group p31m is a wallpaper group.
\end{lemma}

\begin{lemma}[p6 is a wallpaper group]
    \label{lem:p6_is_wallpaper}
    \lean{WallpaperGroup.p6.isWallpaperGroup}
    \uses{def:p6, def:wallpaper_group}
    \notready
    The group p6 is a wallpaper group.
\end{lemma}

\begin{lemma}[p6m is a wallpaper group]
    \label{lem:p6m_is_wallpaper}
    \lean{WallpaperGroup.p6m.isWallpaperGroup}
    \uses{def:p6m, def:wallpaper_group}
    \notready
    The group p6m is a wallpaper group.
\end{lemma}

%%%%%%%%%%%%%%%%%%%%%%%%%%%%%%%%%%%%%%%%%%%%%%%%%%%%%
\section{Distinctness}
%%%%%%%%%%%%%%%%%%%%%%%%%%%%%%%%%%%%%%%%%%%%%%%%%%%%%

\begin{definition}[Wallpaper group invariants]
    \label{def:invariants}
    \lean{WallpaperGroupInvariants}
    \uses{def:wallpaper_point_group, def:wallpaper_translations, def:symmorphic}
    \notready
    The following are isomorphism invariants of a wallpaper group:
    \begin{enumerate}
        \item The isomorphism class of the point group $\PG(\Gamma)$
        \item The lattice type (determined by the symmetry group of $\TG(\Gamma)$)
        \item Whether $\Gamma$ is symmorphic
        \item The location of rotation centers relative to the lattice
        \item The orientation of reflection axes relative to the lattice
    \end{enumerate}
\end{definition}

\begin{lemma}[The 17 groups are pairwise distinct]
    \label{lem:distinct}
    \lean{wallpaperGroups_distinct}
    \uses{def:invariants, def:p1, def:p2, def:pm, def:pg, def:cm, def:pmm, def:pmg, def:pgg, def:cmm, def:p4, def:p4m, def:p4g, def:p3, def:p3m1, def:p31m, def:p6, def:p6m}
    \notready
    The 17 wallpaper groups p1, p2, pm, pg, cm, pmm, pmg, pgg, cmm, p4, p4m, p4g, p3, p3m1, p31m, p6, p6m are pairwise non-isomorphic.
\end{lemma}

\begin{proof}
    \notready
    Distinguish groups by their invariants:
    \begin{itemize}
        \item Point group order distinguishes many: orders 1, 2, 3, 4, 6, 8, 12
        \item Within same point group: lattice type or symmorphic property differs
        \item p3m1 vs p31m: both have $\Dih{3}$ and hexagonal lattice, but differ in whether 3-fold centers lie on reflection axes
    \end{itemize}
\end{proof}

%%%%%%%%%%%%%%%%%%%%%%%%%%%%%%%%%%%%%%%%%%%%%%%%%%%%%
\section{Completeness by Case Exhaustion}
%%%%%%%%%%%%%%%%%%%%%%%%%%%%%%%%%%%%%%%%%%%%%%%%%%%%%

\begin{lemma}[Oblique lattice completeness]
    \label{lem:oblique_complete}
    \lean{obliqueLattice_wallpaperGroups}
    \uses{def:oblique_lattice, lem:compatible_point_groups, lem:wallpaper_ses}
    \notready
    The only wallpaper groups with an oblique lattice are p1 and p2.
\end{lemma}

\begin{proof}
    \uses{lem:p1_is_wallpaper, lem:p2_is_wallpaper}
    \notready
    The only point groups compatible with an oblique lattice are $\Cyc{1}$ and $\Cyc{2}$. Both are cyclic, so the short exact sequence always splits. This gives exactly p1 and p2.
\end{proof}

\begin{lemma}[Rectangular lattice completeness]
    \label{lem:rectangular_complete}
    \lean{rectangularLattice_wallpaperGroups}
    \uses{def:rectangular_lattice, lem:compatible_point_groups}
    \notready
    The wallpaper groups with a rectangular lattice are pm, pg, pmm, pmg, pgg.
\end{lemma}

\begin{proof}
    \uses{lem:pm_is_wallpaper, lem:pg_is_wallpaper, lem:pmm_is_wallpaper, lem:pmg_is_wallpaper, lem:pgg_is_wallpaper, lem:non_symmorphic_glide}
    \notready
    Compatible point groups: $\Cyc{1}, \Cyc{2}, \Dih{1}, \Dih{2}$.
    \begin{itemize}
        \item $\Cyc{1}$: gives p1 (but that's oblique, so excluded)
        \item $\Cyc{2}$: gives p2 (also oblique type)
        \item $\Dih{1}$: split → pm, non-split → pg
        \item $\Dih{2}$: split → pmm, one direction non-split → pmg, both non-split → pgg
    \end{itemize}
\end{proof}

\begin{lemma}[Centered rectangular lattice completeness]
    \label{lem:centered_complete}
    \lean{centeredRectangularLattice_wallpaperGroups}
    \uses{def:centered_rectangular_lattice, lem:compatible_point_groups}
    \notready
    The wallpaper groups with a centered rectangular lattice are cm and cmm.
\end{lemma}

\begin{proof}
    \uses{lem:cm_is_wallpaper, lem:cmm_is_wallpaper}
    \notready
    Compatible point groups: $\Dih{1}, \Dih{2}$ (excluding $\Cyc{1}, \Cyc{2}$ which give oblique-type groups). The centering forces the extension to split, giving cm and cmm.
\end{proof}

\begin{lemma}[Square lattice completeness]
    \label{lem:square_complete}
    \lean{squareLattice_wallpaperGroups}
    \uses{def:square_lattice, lem:compatible_point_groups}
    \notready
    The wallpaper groups with a square lattice are p4, p4m, p4g.
\end{lemma}

\begin{proof}
    \uses{lem:p4_is_wallpaper, lem:p4m_is_wallpaper, lem:p4g_is_wallpaper}
    \notready
    Only $\Cyc{4}$ and $\Dih{4}$ require a square lattice.
    \begin{itemize}
        \item $\Cyc{4}$: always splits → p4
        \item $\Dih{4}$: split → p4m, non-split → p4g
    \end{itemize}
\end{proof}

\begin{lemma}[Hexagonal lattice completeness]
    \label{lem:hexagonal_complete}
    \lean{hexagonalLattice_wallpaperGroups}
    \uses{def:hexagonal_lattice, lem:compatible_point_groups}
    \notready
    The wallpaper groups with a hexagonal lattice are p3, p3m1, p31m, p6, p6m.
\end{lemma}

\begin{proof}
    \uses{lem:p3_is_wallpaper, lem:p3m1_is_wallpaper, lem:p31m_is_wallpaper, lem:p6_is_wallpaper, lem:p6m_is_wallpaper}
    \notready
    Point groups $\Cyc{3}, \Cyc{6}, \Dih{3}, \Dih{6}$ require hexagonal lattice.
    \begin{itemize}
        \item $\Cyc{3}$: splits → p3
        \item $\Cyc{6}$: splits → p6
        \item $\Dih{3}$: splits, but two distinct orientations → p3m1 and p31m
        \item $\Dih{6}$: splits → p6m
    \end{itemize}
\end{proof}

%%%%%%%%%%%%%%%%%%%%%%%%%%%%%%%%%%%%%%%%%%%%%%%%%%%%%
\section{Main Theorem}
%%%%%%%%%%%%%%%%%%%%%%%%%%%%%%%%%%%%%%%%%%%%%%%%%%%%%

\begin{theorem}[Classification of wallpaper groups]
    \label{thm:classification}
    \lean{wallpaper_classification}
    \uses{def:wallpaper_group, lem:distinct, lem:oblique_complete, lem:rectangular_complete, lem:centered_complete, lem:square_complete, lem:hexagonal_complete, lem:point_group_crystallographic, thm:bravais_classification}
    \notready
    Every wallpaper group is isomorphic to exactly one of the following 17 groups:
    \[
        \text{p1, p2, pm, pg, cm, pmm, pmg, pgg, cmm, p4, p4m, p4g, p3, p3m1, p31m, p6, p6m.}
    \]
\end{theorem}

\begin{proof}
    \notready
    \textbf{Existence}: Lemmas~\ref{lem:p1_is_wallpaper}--\ref{lem:p6m_is_wallpaper} show each of the 17 is a wallpaper group.

    \textbf{Distinctness}: Lemma~\ref{lem:distinct}.

    \textbf{Completeness}: Let $\Gamma$ be any wallpaper group. By Lemma~\ref{lem:point_group_crystallographic}, $\PG(\Gamma)$ is one of the 10 crystallographic point groups. By Theorem~\ref{thm:bravais_classification}, $\TG(\Gamma)$ is one of 5 lattice types. The compatibility constraints (Lemma~\ref{lem:compatible_point_groups}) and extension enumeration (Lemmas~\ref{lem:oblique_complete}--\ref{lem:hexagonal_complete}) show $\Gamma$ is isomorphic to one of the 17.
\end{proof}

\begin{corollary}[There are exactly 17 wallpaper groups]
    \label{cor:seventeen}
    \lean{wallpaper_count}
    \uses{thm:classification}
    \notready
    Up to isomorphism, there are exactly 17 wallpaper groups.
\end{corollary}



\appendix

\chapter{Mathlib4 Dependencies}

This project relies on the following Mathlib4 infrastructure:

\section{Already Available}
\begin{itemize}
    \item \texttt{EuclideanSpace $\R$ (Fin 2)} --- Euclidean plane
    \item \texttt{LinearIsometryEquiv} --- Orthogonal group elements
    \item \texttt{SemidirectProduct} --- Semidirect product construction
    \item \texttt{IsZLattice} --- $\Z$-lattice structure
    \item \texttt{DihedralGroup n} --- Abstract dihedral groups
    \item \texttt{ZMod n} --- Cyclic groups
    \item \texttt{Matrix.trace} --- Matrix trace
\end{itemize}

\section{To Be Constructed}
\begin{itemize}
    \item $\Euc(2)$ as explicit semidirect product $\R^2 \sdp \Orth(2)$
    \item Embedding of \texttt{DihedralGroup n} into $\Orth(2)$
    \item Discrete subgroup predicate for topological groups
    \item Cocompact subgroup predicate
    \item Classification of finite subgroups of $\Orth(2)$
    \item The 5 Bravais lattice types
    \item The 17 wallpaper groups
\end{itemize}

\chapter{Open Questions for Discussion}

\begin{enumerate}
    \item \textbf{$\Euc(2)$ construction}: Is there existing Mathlib infrastructure for affine isometry groups as semidirect products?

    \item \textbf{Discrete subgroups}: What's the best way to express ``discrete subgroup of a topological group'' in Mathlib4?

    \item \textbf{Cocompactness}: Is there existing API for cocompact group actions or compact quotients?

    \item \textbf{$\Orth(2)$ classification}: Has anyone formalized that finite subgroups of $\Orth(2)$ are cyclic or dihedral?

    \item \textbf{DihedralGroup embedding}: What's the cleanest way to embed \texttt{DihedralGroup n} into \texttt{LinearIsometryEquiv}?

    \item \textbf{Lattice classification}: Is the classification of 2D lattices into 5 Bravais types anywhere in Mathlib?

    \item \textbf{Related projects}: Are there any related formalizations (crystallographic groups, space groups) in progress?
\end{enumerate}

\chapter{References}

\begin{enumerate}
    \item Armstrong, M.A. \textit{Groups and Symmetry}. Springer UTM, 1988.
    \item Schwarzenberger, R.L.E. ``The 17 Plane Symmetry Groups.'' \textit{Mathematical Gazette}, 1974.
    \item Conway, J.H., Burgiel, H., and Goodman-Strauss, C. \textit{The Symmetries of Things}. A K Peters, 2008.
    \item International Tables for Crystallography, Volume A.
\end{enumerate}

\end{document}
